\chapter{Pseudocode of photoemission functions}\label{sec:appendixA}


\section*{Identification of sunlit surface nodes}
\begin{minipage}{\linewidth}
\begin{lstlisting}[language=C, caption={Pseudo code for finding sunlit object surface nodes},
label={lst:sunlitNodes}]
void oFindAllSunlitNodes(...){

    ... define and initialize variables ...

    for(int k = 0; k<size[z]; k++){
    
        for(int j = 0; j<size[y]; j++){
                
            int flag = 0;
            for(int i = 0; i<size[x]; i++){
                
                index = i*sizeProd[1] + j*sizeProd[2] + k*sizeProd[3];
                for(int n = 0; n < nSurfaceNodes; n++){
                    
                    surfNode = surface[n];
                    if(surfNode == index){
                        ... append index to exposed node array ...
                        flag = 1;
                        break;
                    }
                    
                if(flag) break;
                
                }
            }
        }
    }
}
\end{lstlisting}
\end{minipage}

\section*{Identification of cells adjacent to sunlit spacecraft surfaces}
\begin{minipage}{\linewidth}
    \begin{lstlisting}[language=C, caption={C pseudocode for finding cells adjacent
    to sunlit object surfaces},
    label={lst:sunlitCells}]
    
    void oCellFillingNodes(...){
        
        ... variable definitions and initialization ...
    
        for(long int i = surfaceOffset[a]; i<surfaceOffset[a+1]; i++){
                    
            int nObjNode = 0;
            int yzPlane = 0;        
          
            //cell nodes
            long int p = surface[i];
            long int pu = p + sizeProd[3];
            long int plu = p + sizeProd[2] + sizeProd[3];
            long int pl = p + sizeProd[2]; 
            long int po = p - sizeProd[1];
            long int pou = p - sizeProd[1] + sizeProd[3];
            long int poul = p - sizeProd[1] + sizeProd[3] + sizeProd[2];
            long int pol = p - sizeProd[1] + sizeProd[2];
        
            for(int j = interiorOffset[a]; j < interiorOffset[a+1]; j++){
                long int intNode = interior[j];
                if(po == intNode || pou == intNode ||
                   poul == intNode || pol == intNode){
                    nObjNode++;
                }
            }
        
            for(int j = surfaceOffset[a]; j < surfaceOffset[a+1]; j++){
                long int surfNode = surface[j];
                if(po == surfNode || pou == surfNode || 
                   poul == surfNode || pol == surfNode){
                    nObjNode++;
                }
        
                if(p == surfNode || pu == surfNode || 
                   plu == surfNode || pl == surfNode){
                    yzPlane++;
                }
            }
        
            if(yzPlane == 4 && nObjNode < 8){
                ... store index p as an emitting node for cell filling ...
            }
        }
    }
    \end{lstlisting}
\end{minipage}

\section*{Numerical integration of Planck's blackbody radiation law}
\begin{minipage}{\linewidth}
\begin{lstlisting}[language=C, caption={Numerical integration of Planck's law in terms of photon flux}, label={lst:planck}]
void oPlanckPhotonIntegral(...){

  ... Variable and constant definitions and initialization ...
    
  for(int a=0; a<nObj; a++){
	//sigma[a] = ((double)sigma[a]) / (Planck * speedOfLight * 100.);
	double c1 = Planck * speedOfLight / Boltzmann;
	double x = c1 * 100. * (double)sigma[a]/temperature;
	double x2 = x * x;

	// decide how many terms of sum are needed
	double iterations = 2.0 + 20.0/x;
	iterations = (iterations < 512.) ? iterations : 512;
	int iter = (int)(iterations);
	// add up terms of sum
	double sum = 0.;
	for (int n=1; n<iter; n++) {
	  double dn = 1.0 / n;
	  sum += exp(-n*x) * (x2 + 2.0*(x + dn)*dn) * dn;
	}
	//result, in units of photons/s/m2/sr, convert to photons/timestep
	double kTohc = (Boltzmann * (double)temperature) / (Planck * speedOfLight);
	double solidAngle = (double)area[a] / pow(distFromSun,2.);
	radiance[a] = 2.0 * pow(kTohc,3.) * speedOfLight;
	radiance[a] = radiance[a] * sum;
	radiance[a] *= solidAngle * sunSurfaceArea;
	radiance[a] *= (1.0 - reflectance) * phYield;
	radiance[a] *= time;
  }
    
  ... Store radiance to object structure here ...

}
\end{lstlisting}
\end{minipage}

Listing \ref{lst:planck} shows a numeric implementation of equation \eqref{eq:PlanckX} and equation \eqref{eq:PlanckSum}. The original algorithm \parencite{Blackbody} has been converted from C++ code to C for compatibility with the PINC framework.


\section*{Particle injection into cells adjacent to sunlit spacecraft surfaces}
\begin{minipage}{\linewidth}
\begin{lstlisting}[language=C, caption={C style pseudocode of cell filling algorithm},
                    label={lst:cellInject}]
void pPhotoelectrons(...){

  ... initialize variables ...
  ... divide photoelectron flux by total number of cells to be filled ...

  for(int a=0; a < nObjects; a++){
    for(long int j=0; j<(long)flux[a]; j++){
      for(int i = 0; i < nEmittingNodes; i++){

        long int node = emittingNodes[emittingOffset[a] + i];
        gNodeToGrid3D(... node, pos ... ); //computes node location in grid
        memcpy(newPos, pos, pop->nDims * sizeof(*newPos));
        while(vel[0] >= 0.0){
          vel[0] = univariate_gaussian(... avgVel[a] ...); //normal velocity component
        }
        bivariate_gaussian( ...avgVel[a], &vel[1], &vel[2] ...); //tangential velocity components
        newPos[0] -= random(...); //subtract a random value in range [0,1]
        newPos[1] += random(...);
        newPos[2] += random(...);
        pNewParticle(...); //create the new particle
        ... reset "vel" array to zeroes ...
      }
    }
  }   
}
\end{lstlisting}
\end{minipage}
