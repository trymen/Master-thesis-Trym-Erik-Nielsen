\chapter{Future work}
\label{sec:future}

\section{Improving the object module}
In order to simulate the charging of objects in ambient plasma, PINC uses the capacitance matrix method as described in \parencite{Miyake2009}. The implementation of this method in the PINC framework when the capacitance matrix is applied and charges are redistributed to the object surface assumes the object is composed of a homogeneous, and perfectly conducting, material: Charges due to impinging particles, and photoelectrons, are redistributed evenly in such a way as to achieve an equipotential. Since dielectric surfaces are poor electrical conductors, charged particles tend to remain where they strike on the surface.

In physical systems, this assumption of a perfectly conducting body does not necessarily hold, spacecraft that are composed of both dielectric and conducting surfaces can charge to different levels in sunlight \parencite{LAI2019}. For spacecraft with different surface material composition, this difference in surface potential leads to differential charging. For spacecraft exposed to sunlight, like the MMO, the shadowed dielectric surfaces emit secondary electrons when high temperature electrons impinge of the surface. These secondary electrons are re-absorbed by conducting surfaces grounded to the less negatively charged sunlit side, thereby charging the spacecraft more (in the negative sense) as compared to a perfectly conducting spacecraft. 

Dielectric surfaces could be implemented by splitting the arrays containing surface nodes into separate arrays for conducting surfaces and dielectric surface nodes. Alternatively, since the object module in PINC is able to simulate several objects at the same time, the spacecraft could be divided into separate objects with an additional label marking the object as either dielectric or conducting.


\section{Implementation of additional charging currents}
The photoelectron current is only one of several currents that have a significant impact on the floating potential, and plasma sheath formed around a spacecraft in a drifting plasma. Secondary electrons, discussed in \cref{sec:theory}, are especially important for spacecraft situated close to the sun where the plasma tends to be hotter. They affect both the floating potential, and depth of the potential barriers \parencite{Deca2013}. The current due to backscattered electrons is another such important current that can in some cases exceed the current due incoming fluxes \parencite{Garrett1981}. These currents, in addition to the photoemission current, could be included as a separate "extra current" module where the user of PINC could specify which current they want to be active as part of the input file.

\section{Improving the photoemission current implementation}
Several assumptions were made in how the photoemission current was implemented for this thesis. These assumptions can be addressed to improve the accuracy of the model. First, we assumed in chapter \cref{sec:methods} that the incident angle of the sun to be perpendicular to sunlit surfaces. Reflectance, and thus photoelectric yield, are dependent on the incident angle of photons. As a first approximation the reduction in radiance can be modeled, using Lambert's cosine law, as proportional to the cosine of the incident angle. 

Furthermore, the photoelectron cell injection algorithm is effective but inefficient: By necessity during implementation and debugging, several computations in the function are calculated each time the function is called. Extracting these computations into a structure variable could significantly reduce overall computation times for longer simulations. 

Domain decomposition with the photoemission functionality can also be improved, it was mentioned briefly in section \cref{sec:methods} that splitting the computational domain perpendicularly to the sun-spacecraft axis is not currently possible. If the split of the domain passes through the object to be simulated, another "sunlit" surface is exposed to the function that finds the nodes exposed to sunlight. If this function is rewritten to discard these cut surfaces, then domain decomposition becomes much more streamlined and the code can be more effectively run on a higher number of CPU's.

finally, the photoelectron emission code could be improved for the end user experience by separating the photoelectrons as a separate specie of the same charge and mass as the ambient electrons. PINC supports multi-species plasma, but as for now, the photoelectron injection code does not distinguish between ambient electron and photoelectron.