\chapter{Future work}
\label{sec:future}

\section{Improving the object module}
In order to simulate the charging of objects in ambient plasma, PINC uses the capacitance matrix method as described in \parencite{Miyake2009}. The implementation of this method in the PINC framework when the capacitance matrix is applied and charges are redistributed to the object surface assumes the object is composed of a homogeneous, and perfectly conducting, material: Charges due to impinging particles, and photoelectrons, are redistributed evenly in such a way as to achieve an equipotential. For dielectric surfaces charged particles tend to remain where they struck the surface by definition.

In physical systems, this assumption does not necessarily hold, spacecraft that are composed of both dielectric and conducting surfaces can charge to different levels in sunlight \parencite{LAI2019}. For spacecraft with different surface material composition, this difference in surface potential leads to differential charging. For spacecraft exposed to sunlight, like the MMO, the shadowed dielectric surfaces emit secondary electrons when high temperature electrons impinge of the surface. These secondary electrons are re-absorbed by conducting surfaces grounded to the less negatively charged sunlit side, thereby charging the spacecraft more (in the negative sense) as compared to a perfectly conducting spacecraft. 

Dielectric surfaces could be implemented by separating the arrays containing surface nodes into separate arrays for conducting surfaces and dielectric surface nodes. Alternatively, since the object module in PINC is able to simulate several objects at the same time, the spacecraft could be divided into separate objects labeled as either dielectric or conducting.


\section{Implementation of additional charging currents}


\section{Improving the photoemission current implementation}
Several assumptions were made in how the photoemission current was implemented for this thesis. These assumptions can be addressed to improve the accuracy of the model. First, we assumed in chapter \cref{sec:methods} that the incident angle of the sun to be perpendicular to sunlit surfaces. Reflectance, and thus photoelectric yield, are dependent on the incident angle of photons. As a first approximation, the reduction in radiance can be modeled as proportional to the cosine of the incident angle. 