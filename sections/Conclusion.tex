\chapter{Summary and Discussion}\label{sec:conclusion}

\section{Photoemission implementation}
The main objective of this thesis was to analyze the charging of the MMO spacecraft in direct sunlight while in orbit around Mercury using three dimensional Particle-In-Cell simulations. Achieving this goal necessitated further developing the object charging module of the PIC framework PINC with functionality for including a photoemission current.

In order to inject photoelectrons in a physically realistic way, the new code had to meet the following requirements: The code had to be able to identify what surfaces of the simulated spacecraft were exposed to sunlight, the average temperature and the amount of photoelectrons to inject, and crucially, how to inject the photoelectrons from the identified photoemissive surfaces.

Two methods were implemented for determining the average photoelectron temperature and what amount to inject per timestep. The simplest method implemented took the photoelectron temperature and current density as input variables, this method then converted the current density to the correct number of photoelectrons and gave these photoelectrons a Maxwellian velocity distribution using the average temperature given by the user. The second method, and the one used to simulate the MMO in \cref{sec:results}, computed the amount of photoelectrons and average temperature directly by numerically integrating Planck's law for Blackbody radiation using the photoelectron yield and work function of the spacecraft surface, as well as the spacecrafts distance from the sun as input parameters. 

Three methods for injecting photoelectrons were implemented and compared to other PIC codes for verification purposes: The method selected sampled the tangential and normal velocity of the photoelectrons with respect to the sunlit surface, and filling the computational cells adjacent to the sunlit spacecraft uniformly. This method was found to be the injection method that was best able to reproduce the floating potential other PIC codes converged to. We were also able to show that the algorithm selected for photoelectron injection has a significant impact on the charging behaviour and the floating potential the simulated spacecraft will converge to.  


\section{Comparison of results with theory and other charging simulations}

For any spacecraft with plasma instruments as a payload, the floating potential is a critical value that must be considered. Spacecraft charged to a non-zero potential will attract particles with charge of opposite sign to the potential, and reflect particles with charge of the same sign as the potential. This causes a sheath to form around the spacecraft where the density of the attracted particles is greater than the ambient plasma. To accurately measure the ambient plasma temperature and density, the plasma instruments must therefore be able to account for this difference in density.

In section \cref{sec:results}, we have demonstrated that the MMO spacecraft will charge to a positive floating potential in low density plasma due to a high ratio between the photoelectron current and current due to impinging electrons from the ambient plasma, this is in accordance with data from \parencite{Garrett1981}, for spacecraft charging in solar wind plasmas. We have also demonstrated that the floating potential of a spacecraft will increase when booms are deployed and not electrically isolated. Furthermore, \ref{tab:MMOresultsSummary} show that the floating potential of a spacecraft is unaffected even in the presence of weak magnetic fields, or slow relative plasma drift relative to the spacecraft: Assuming a non-drifting and un-magnetized plasma, OML theory and the current balance equation show that the floating potential is dependent on the density, temperature of ambient electron and photoelectrons, as well as the shape of the spacecraft. In our simulations these parameters are kept constant when the relative drift velocity is changed, or when an external magnetic field is included, and the floating potential remains constant. This can also be seen when photoemission is not included, for simulation cases 3 and 4, where the floating potential is described by equation \eqref{eq:noPhPhif}. Similarly for cases 9 and 10 in table \ref{tab:MMOexperiments}, there is a large decrease in the floating potential of the MMO when the photoelectron temperature is increased. When the photoelectron temperature increases, more electrons  This is again in accordance with OML theory, and can be seen in PIC simulations of other spacecraft as well \parencite{Sjogren2012}. 

For all the simulations cases in \cref{sec:results}, we have demonstrated the presence of a photoelectron cloud. When a potential profile is plotted as a line passing through the spacecraft and parallel to the sun-spacecraft axis, a significant decrease in the potential relative to the ambient plasma is seen. This evidence of a potential barrier forming due to the photoelectron cloud present in from of spacecraft surfaces adjacent to the sunlit spacecraft surfaces. Such barriers are also demonstrated in other PIC spacecraft charging simulations \parencite{Meyer-Vernet2007, Sjogren2012, Deca2013}. We also see a dependence of the potential barrier height on the photoelectron temperature, which is demonstrated mathematically in the paper by Zhao et al. \parencite{Zhao1996}. However, the panel perpendicular to the sun-spacecraft vector shows variations in the potential barrier. None of the simulations we have compared our simulations to show this phenomena. \cref{sec:appendixD} contains a short discussion on these variations. We were not able to show if these variations are due to the geometrical shape of the MMO or due to numerical errors resulting from the coarse voxelization of the spacecraft or due to the small non-physical electric field internally in the spacecraft.  

The formation of a photoelectron sheath are seen in our results from electron density profile plots. Close to the spacecraft, the electron density reaches maximums of more than a thousand times that of the ambient plasma, with a similar decrease in the density of ions close to the spacecraft. Using equation \eqref{eq:PhSheathThickness}, we were able to estimate the thickness of the photoelectron sheath. However, there is no evidence from our results that show a relationship between the photoelectron temperature and the photoelectron sheath thickness, which is not correct \parencite{Zhao1996}. Furthermore, within the photoelectron sheath, our results show spikes in potential along the booms. Literature shows these spikes can occur for booms that are electrically isolated from the main body of the spacecraft \parencite{LAI2019, Paulsson2019, Miyake2013}. The MMO was simulated as perfectly conducting however, and we were unable to show if these spikes are physical or due to a numerical error. 

% -floating potential
% -boom effects
% -potential barrier
% -sheath asymmetry 
% -PH temp variations