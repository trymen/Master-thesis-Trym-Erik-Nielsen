\chapter{Results}
\label{sec:results}

The charging behaviour of a spacecraft is dependent on several factors; the material composition of the craft, its dimensions, its location in space, the local time, and the space weather \parencite{LAI2019} \todo{Update reference, bib file contains extract of the book}. In this thesis, I wish to investigate what effects the presence of MMO's booms have on its charging behaviour, additionally I wished to investigate whether the direction of the ambient plasma drift or the photoelectron temperature would significantly change how the spacecraft charges.

The booms on the MMO extend the characteristic length of the spacecraft to be longer than that of the Debye length of the plasma found in table \ref{tab:PlasmaParamMMO}, as such I expect there to be a difference in the thickness of the plasma sheaths formed. 

Varying the direction of drift may impact the convection of photoelectron away from the surface of the sunlit surface of the spacecraft, which could potentially lead to a difference in the floating potential. Varying the average energy of the emitted photoelectrons will change the charging behaviour of the MMO, as seen in the Rosetta charging simulation paper \parencite{Sjogren2012}. To observe to what extent the floating potential and the thickness of the plasma sheath changes is the goal of varying the photoelectron temperature. 

\begin{center}
\begin{table}[H]
%\centering
\begin{tabular}{p{1.5cm}|p{1.5cm}|p{1.5cm}|p{1.5cm}|p{1.5cm}|p{1.5cm}|p{1.5cm}}
\toprule
\toprule
 & No photoelectrons & drift along $+X$ axis & drift along $-Z$ axis & Inclusion of booms & Mercury magnetic field & Photoelectron temp. $3 \; eV$ \\
\hline
Case 1 & \text{X} & \text{X} & & & &\\
\hline
Case 2 & & \text{X} & & & &\\
\hline
Case 3 & & & \text{X} & & &\\
\hline
Case 4 & & & \text{X} & & \text{X} &\\
\hline
Case 5 & & & \text{X} & & & \text{X}\\
\hline
Case 6 & \text{X} & \text{X} & & \text{X} & &\\
\hline
Case 7 & & \text{X} & & \text{X} & &\\
\hline
Case 8 & & & \text{X} & \text{X} & &\\
\hline
Case 9 & & & \text{X} & \text{X} & \text{X} &\\
\hline
Case 10 & & & \text{X} & \text{X} & & \text{X}\\
\bottomrule
\bottomrule
\end{tabular}
\caption{Summary of the numerical experiments of the MMO in orbit around Mercury, carried out with PINC}
\label{tab:MMOexperiments}
\end{table}
\end{center}

Table \ref{tab:MMOexperiments} gives an overview of the numerical simulations of the MMO spacecraft carried out for this thesis, case 1 and case 5 give a baseline of the charging behaviour where no photoemission is included and serves as a point of comparison between the photoelectron current and the ambient electron plasma current the MMO is subjected to. 

Cases 1 to 5, and cases 5 to 10 are equivalent but for the inclusion of the booms, with case 9 being the closest to the true conditions the MMO will experience when its orbit passes over the north cusp (ecliptic north) of Mercury. As such, case 8 is used as the baseline for the simulation in which the photoelectron temperature is varied from the value computed by PINC.

Save for cases 1 and 6, the simulations without photoemission, the computational domain was decomposed into 64 sub-domains for parallel processing. Cases 1 and 6 were decomposed into 128 sub-domains, the reason being that in the cases containing photoemission, the domain was not divided along the x-axis thus splitting the spacecraft in the YZ plane. This would account for another surface interior to the spacecraft that would have to be discarded as a source for photoemissive nodes.

For the sake of brevity, we will refer to the case number as shown in table \ref{tab:MMOexperiments} when presenting analyses for these five experiments. Results will be presented in order such that the plasma conditions closest to those the MMO will actually experience when orbiting Mercury is presented as the last simulation where photoemission parameters are computed directly. Photoemission with a different electron temperature is also presented, since an assumption of constant photoelectron yield was made, the impact of a higher photoelectron temperature will be compared to the other experiments in which the average photoelectron temperature was computed using Planck's law. We have also chosen to group simulations with and without booms together to make comparison across the two configurations simpler.

Except for a computational charging analysis of the combined BepiColombo spacecraft under thrust (using the MTM, the Mercury Transfer Module) \insertref{ESA Bepi sim} we were unable to find similar numerical experiments of the charging of the MMO spacecraft, we will therefore compare our results to theory through current balance analysis as presented in \ref{sec:theory}. We will also compare our results qualitatively to numerical charging experiments for different plasma parameters and different objects, such as in the paper by Deca et al. \parencite{Deca2013} used for our verification simulations.
\todo{Add sheath thickness analysis}
From current balance, with photoemission included, we expect the variation of drift direction (comparing cases 2 and 3 to cases 7 and 8) to have minimal impact on the floating potential PINC will converge to. Similarly we expect the sheath thickness and sheath plasma densities to be comparable for different drift directions. Across all five simulation types run, we expect the floating potential and sheath thickness to be larger in magnitude when the booms on the MMO are fully extended; since the booms are modeled as photoemissive, the greater surface area will lead to an overall higher electron current flowing from the spacecraft leading to a higher floating potential \todo{That can't be right...}. 

The inclusion of the external magnetic field of Mercury could potentially impact the fraction of photoelectrons with a greater than average temperature being able to leave the surface of the spacecraft, if the particles' velocity vector does not align with the magnetic field the force exerted by the magnetic field could trap electrons more efficiently, leading to a lower floating potential.        

%FLOATING POTENTIAL SUMMARY
% \begin{table}[]
% \centering
% \begin{tabular}{cccccc}
% \hline
% & No photoelectron & Drift along Z & Drift along X & External Magnetic Field & Photoelectron temperature 3 eV\\\hline
% With booms & N/A & 105.419 & 105.400 & 105.400 & 78.154\\\hline
% Without booms & N/A & 100.710 & 100.696 & 100.699 & N/A\\\hline
% \end{tabular}
% \caption{Dummy text}
% \label{tab:FlotingPotSummary}
% \end{table}
\section{Charging without photoemission}\label{subsec:noPHsec}
%FLOATING POTENTIAL CONVERGENCE
\begin{center}
    \begin{figure}[H]
      \begin{subfigure}[b]{0.75\textwidth}
      \includegraphics[width=\columnwidth]{figures/MMO/noPH/WB/C_noPH_WB.png}
      \caption{Booms}
      \label{fig:C_noPH_WB}
    \end{subfigure}
    \par\bigskip
    \begin{subfigure}[b]{0.75\textwidth}
      \includegraphics[width=\columnwidth]{figures/MMO/noPH/NB/C_noPH_NB.png}
      \caption{Without booms}
      \label{fig:C_noPH_NB}
    \end{subfigure}
  \caption{Timeseries plot of potential of the MMO with and without booms. The potential has been converted from PINC dimensionless units to Volts. The inset plots shows the potential of the two configurations for last 10,000 timesteps.}
    \label{fig:ConvnoPH}
  \end{figure}
\end{center}

The timeseries in fig \ref{fig:ConvnoPH} plots the potential of the MMO for each timestep simulated. From each inset plot, both for the MMO with booms and without, it is apparent from the slope that the system has not converged to a floating potential. Since the ambient plasma density around the MMO were approximately 100 times less dense than the value used in the verification simulation and no other current than ambient particle charging was present, a larger timestep was used in addition to a longer total simulation time. 

In order to compare our results with theory, we can estimate the floating potential the simulation would have converged to by fitting a function to our data and propagating the function forward in time until a steady state has been reached. Using Hill's function of the form 
\begin{equation*}
    \phi(t) = \phi_f \frac{t}{h + t},
\end{equation*}
where $\phi_f$ is the MMO floating potential without a photoemission current present, we estimate a floating potential of $\phi_f \approx - 283 V$. A full analysis of the curve fitting and the Python script used to fit the curves is given in  \cref{sec:appendixC}.

From chapter \cref{sec:theory}, we know that we can estimate the floating potential from OML theory and a current balance equation. However, one of the assumptions of OML theory is that the plasma must be non-drifting and Maxwellian \parencite{Shipple1981}. For non drifting plasmas, the thermal speed of the charged particles is significantly larger than the relative velocity of a spacecraft to the plasma \parencite{Jacobsen2010}. For our simulations, we assume a drift speed of $100 km s^{-1}$. From table \ref{tab:PlasmaParamMMO} the magnetospheric plasma has an electron temperature of $T_e = 1.16 \times 10^6$ which amounts to a thermal speed of $4193 km s^{-1}$. Since the electron thermal speed is so high compared to our plasma drift, we can safely make the approximation that electrons in the plasma are non-drifting. 

For cases 1 and 6, the only currents affecting the spacecraft potential are the currents due to impinging ions and electron. The current balance equation is therefore
\begin{equation}\label{eq:noPHCurrentBal}
    I_i(\phi_f) - I_e(\phi_f) = 0,
\end{equation}
where $I_i$ is the current due to impinging ions (proton), and $I_e$ is the current due to impinging electrons. The main body of the MMO can be approximated as a cylinder of height $0.9 \; m$ and with a diameter of $1.8 \; m$. We can then substitute in equation \eqref{eq:ICylinder}, and \eqref{eq:Irepelled} to get
\begin{equation}\label{eq:noPHIinter}
  I_i(0)\left(\frac{2}{\sqrt{\pi}} \sqrt{1 -  \frac{e \phi_f}{k T_i}}\right) -  I_e(0)\exp(\frac{e \phi_f}{k T_e}) = 0.
\end{equation}
The current when the spacecraft is at zero potential for a cylinder is given by \parencite{LAI2019}
\begin{equation}\label{eq:IthCyl}
    I_s(0) = 2 \pi a n_s q_s v L,
\end{equation}
where $a$ is the cylinder radius, and $L$ is the cylinder length and $v$ is the speed of the charged particle an infinite distance from the cylinder. Substituting equation \eqref{eq:IthCyl} into equation \eqref{eq:noPHIinter}, we have
\begin{equation}\label{eq:noPhPhif}
    \frac{2}{\sqrt{\pi}} \sqrt{1 -  \frac{e \phi_f}{k T_i}} = - \exp(\frac{e \phi_f}{k T_e}).
\end{equation}
Here we have assumed that the floating potential of the MMO will be negative, and that $\eta_s = \exp(- \frac{q_s \phi_f}{k T_s}) > 2$. We solve the transcendental equation \eqref{eq:noPhPhif} by plugging in the known values for $T_i$ and $T_e$ to get a floating potential $\phi_f = -298.3 V$.
%POTENTIAL THROUGH CENTER OF OBJECT
\begin{center}
    \begin{figure}[H]
      \begin{subfigure}[b]{0.61\textwidth}
      \includegraphics[width=\textwidth]{figures/MMO/noPH/WB/L_noPH_WB.png}
      \caption{Booms}
      \label{fig:L_noPH_WB}
    \end{subfigure}
    \begin{subfigure}[b]{0.61\textwidth}
      \includegraphics[width=\textwidth]{figures/MMO/noPH/NB/L_noPH_NB.png}
      \caption{Without booms}
      \label{fig:L_noPH_NB}
    \end{subfigure}
  \caption{\ref{fig:L_noPH_NB} and \ref{fig:L_noPH_WB} show a potential profile along the X axis for the MMO without and with booms respectively. The line is plotted at $(x,y) = (13.95 m, 13.95 m)$, or node points $(x,y) = (62,62)$, and passes through the main octagonal body of the spacecraft. The X axis units are in number of nodes from the origin.}
\label{fig:Line_noPH_comb}
  \end{figure}
\end{center}

Even without PINC converging to a floating potential, we can still make comparisons of the plasma sheath formed around the spacecraft by plotting the potential profile of the domain. Figure \ref{fig:Line_noPH_comb}, shows the potential relative to the ambient plasma plotted on a center line passing through the spacecraft octagonal body. comparing figure \ref{fig:L_noPH_WB} with figure \ref{fig:L_noPH_NB}, we see a difference in sheath thickness. For a negatively charged probe, we can estimate the thickness of the sheath using Bohm's sheath criterion \parencite{Chen2018}:
\begin{equation}\label{eq:BohmCrit}
    u_0 \approx \left(\frac{K T_e}{m_i}\right)^{1/2},
\end{equation}
where $u_0$ is the velocity required by ions at the sheath boundary. For ions to be accelerated to this velocity we can estimate the required sheath edge potential as
\begin{equation}\label{eq:PhiSheathEdge}
    \phi_s = -\frac{1}{2} \frac{K T_e}{e}.
\end{equation}

Where $\phi_s$ is the potential relative to the ambient plasma. Setting the electron temperature, $T_e$, to $1.16 \times 10^6$ we find the sheath boundary potential to be approximately $-50 V$ relative to the main plasma body. The sheath thickness is then the distance from the potential along the x axis to where the potential is the same as the sheath boundary potential. From the plots we then estimate a sheath thickness of 2.5 meters for the MMO configuration without booms, and 3.2 meters for the configuration with booms.

The sheath thickness, $d$, can also be computed directly by re-arranging the Child-Langmuir law \parencite{Chen2018}:
\begin{equation}\label{eq:childLangmuir}
    d^2 = \frac{4}{9} \left(\frac{2e}{m_i}\right)^{1/2} \frac{\abs{\phi_w}^{3/2}}{4 \pi j_{i,s}}.
\end{equation}
where $\phi_w$ is the "wall", or spacecraft potential, and $j_{i,s}$ is the ion saturation current density. The ion saturation current density is given by
\begin{equation}
    j_{i,s} = q_e n_e c_s.
\end{equation}
Where $c_s$ is the ion speed of sound, and is computed as \parencite{Paulsson2019}
\begin{equation}\label{eq:ionMach}
    c_s = \sqrt{\left(\frac{K_b T_e}{m_i}\right)},
\end{equation}
Plugging in the known spacecraft potential, the electron density, and electron temperature from table \ref{tab:PlasmaParamMMO}  the plasma thickness comes out to be 3.63 meters, which is comparable to our estimates. 

%AVERAGE POTENTIAL ISOLINES XY
\begin{center}
\begin{figure}[H]
  \begin{subfigure}[b]{0.61\textwidth}
    \includegraphics[width=\textwidth]{figures/MMO/noPH/WB/P_noPH_WB.png}
    \caption{Booms}
    \label{fig:P_noPH_WB}
  \end{subfigure}
  \hfill
  \begin{subfigure}[b]{0.61\textwidth}
    \includegraphics[width=\textwidth]{figures/MMO/noPH/NB/P_noPH_NB.png}
    \caption{Without booms}
    \label{fig:P_noPH_NB}
  \end{subfigure}
  \caption{\ref{fig:P_noPH_WB} and \ref{fig:P_noPH_WB} are 2D slices through $Z = 14.4 m$ showing the potential profile of the entire computational domain for simulation cases 1 and 6.}
  \label{fig:PMMOnoPH}
\end{figure}
\end{center}

Figure \ref{fig:PMMOnoPH} shows 2D cut in the plane $Z=14.4$, dividing the spacecraft in half, and is plotted at the last simulated timestep (timestep 40,000). The contours are plotted using the matplotlib Python package, with contour lines subdivided into 500 levels to better distinguish the plasma sheath. Values are converted from normalized PINC internal values to Volts. The color boundary between orange and red approximately marks the plasma sheath edge. We can see this boundary is equidistant from the MMO body in the boomless configuration, whereas the sheath boundary is relatively closer to the boom tips than to the main octagonal body of the spacecraft. 

%PARTICLE DENSITIES (rho_i and rho_e)
%RHO_I
\begin{center}
    \begin{figure}[H]
      \begin{subfigure}[b]{0.61\textwidth}
      \includegraphics[width=\textwidth]{figures/MMO/noPH/WB/I_noPH_WB.png}
      \caption{Booms}
      \label{fig:I_noPH_WB}
    \end{subfigure}
    \begin{subfigure}[b]{0.61\textwidth}
      \includegraphics[width=\textwidth]{figures/MMO/noPH/NB/I_noPH_NB.png}
      \caption{Without booms}
      \label{fig:I_noPH_NB}
    \end{subfigure}
  \caption{Ion density profile plotted at $Z = 14.4 m$, the color gradient is normalized against the ion plasma density from table \ref{tab:PlasmaParamMMO}.}
\label{fig:IonsNoPH}
  \end{figure}
\end{center}

%RHO_E
\begin{center}
    \begin{figure}[H]
      \begin{subfigure}[b]{0.61\textwidth}
      \includegraphics[width=\textwidth]{figures/MMO/noPH/WB/E_noPH_WB.png}
      \caption{Booms}
      \label{fig:E_noPH_WB}
    \end{subfigure}
    \begin{subfigure}[b]{0.61\textwidth}
      \includegraphics[width=\textwidth]{figures/MMO/noPH/NB/E_noPH_NB.png}
      \caption{Without booms}
      \label{fig:E_noPH_NB}
    \end{subfigure}
  \caption{Electron density profile plotted at $Z = 14.4 m$, the color gradient is normalized against the electron plasma density from table \ref{tab:PlasmaParamMMO}.}
  \label{fig:ElectronsNoPH}
  \end{figure}
\end{center}

Figures \ref{fig:IonsNoPH} and \ref{fig:ElectronsNoPH} show the density of ions (protons) and electrons respectively plotted at timestep 400, 1\% through the total simulation time. Similarly to figure \ref{fig:PMMOnoPH}, the 2D slice is placed at $Z = 14.4 m$, dividing the spacecraft in halves. The density values have been normalized against the ambient plasma density, and as before, 500 levels were used for the contour lines drawn. For cases 1 and 6, we set the plasma drift along the positive X axis, \ref{fig:IonsNoPH} clearly shows an ion wake forming "behind" the spacecraft for both boom configurations. We also note the slightly higher density of ions directly in "front" of the spacecraft, denoting the plasma sheath that has formed even this early in the simulation. Figure \ref{fig:ElectronsNoPH} gives the clearest picture of the shape of the plasma sheath formed around the spacecraft, in the MMO configuration with booms, the sheath forms an ellipsoid around the spacecraft where, relative to the ambient plasma, almost no electrons reside. The sheath in the boomless MMO configuration is circular in shape, which gives a good indication that computing the theoretical floating potential of the boomless MMO configuration can be done by assuming the MMO to be a cylinder with the same radius as the circle circumscribing the MMO octagonal body.

\section{Charging with photoemission}

\subsection*{Drift parallel to X axis}

%FLOATING POTENTIAL CONVERGENCE
\begin{figure}[H]
  \centering
  \begin{subfigure}[b]{0.75\textwidth}
  \includegraphics[width=\columnwidth]{figures/MMO/posX/WB/C_posX_WB.png}
  \caption{Booms}
  \label{fig:C_posX_WB}
\end{subfigure}
\hfill
\begin{subfigure}[b]{0.75\textwidth}
  \includegraphics[width=\columnwidth]{figures/MMO/posX/NB/C_posX_NB.png}
  \caption{Without booms}
  \label{fig:C_posX_NB}
\end{subfigure}
\caption{Time series plot of the potential of the two MMO configurations for drift along X axis. The insets plots the same timeseries starting at 1000 timesteps, where the potential of the spacecraft has begun to oscillate about the floating potential.}
\label{fig:Conv_posX}
\end{figure}

Figure \ref{fig:Conv_posX} shows the convergence to the floating potential of both MMO configurations. Unlike the numerical simulations without photoemission, the potential of both configurations of the MMO converge rapidly to a its floating potential. The slope of the increase in potential early on in the simulation points to the high ratio between the photoemission current and the charging from the impinging electrons from the ambient plasma. The floating potential for MMO with booms, as seen in figure \ref{fig:C_posX_WB} overshoots the floating potential by several volts, before converging to 105.4 V. In the boomless configuration, figure \ref{fig:C_posX_NB}, the overshoot is much smaller, and the potential shows oscillatory behaviour around a floating potential of 100.7 V. The overshoot in the floating potential could result from the thin booms extending far from the main body of the MMO developing a potential barrier slower, and smaller in height, than the central body leading to a gradual rather than sudden decrease in the effective photoemission current.

% We can compute the theoretical floating potential of the MMO with booms undeployed with a photoemission current present using the same method as in \cref{subsec:noPHsec}. The current balance equation must include the photoemission current, and therefore takes the form
% \begin{equation}
%     I_i(\phi_f) - I_e(\phi_f) + I_{ph}(\phi_f) = 0.
% \end{equation}
% If we assume a highly positively charged spacecraft, the $I_i{\phi_f}$ term can be neglected, therefore reducing the current balance equation to
% \begin{equation}
%     I_e(\phi_f) = I_e(0)\left(\frac{2}{\sqrt{\pi}} \; \sqrt{\eta_e} \text{erfc}(\sqrt{\eta_e}) \right) \approx I_{ph}.
% \end{equation}
% The photoemission current $I_{ph}$ is dependent on the potential of the spacecraft, since only photoelectrons with trajectories not returning to the spacecraft surface contribute to the spacecraft potential. The photoelectron current is computed as
% \begin{equation}
%     I_{ph} = \frac{N_p \overline{Y}_{ph} q_e}{dt}, 
% \end{equation}
% where $N_p$ is the number of impinging photons per timestep with energy higher than the material work function, and $\overline{Y}_{ph}$ is the averaged photoelectron yield. Plugging in $N_p$ output by the simulation, and $\overline{Y}_{ph}$ from \cref{tab:PlasmaParamMMO}, we have a photoemission current of $I_{ph} = 0.384 A$.


%POTENTIAL THROUGH CENTER OF OBJECT

\begin{figure}[H]
  \begin{subfigure}[b]{0.6\textwidth}
  \includegraphics[width=\textwidth]{figures/MMO/posX/WB/L_posX_WB.png}
  \caption{Booms}
  \label{fig:L_posX_WB}
\end{subfigure}
\begin{subfigure}[b]{0.6\textwidth}
  \includegraphics[width=\textwidth]{figures/MMO/posX/NB/L_posX_NB.png}
  \caption{Without booms}
  \label{fig:L_posX_NB}
\end{subfigure}
\caption{Potential profile along the X axis for the two MMO configurations with drift along the X axis and photoemission included. The line is plotted at $(x,y) = (13.95 m, 13.95 m)$, or node points $(x,y) = (62,62)$, and passes through the main octagonal body of the spacecraft. The X axis units are in number of nodes from the origin. The two values in each plot show the height of the potential barrier formed relative to the ambient plasma.}
\label{fig:Line_posX}
\end{figure}

We next plot the potential along the X axis as seen in figures \ref{fig:L_posX_WB} and \ref{fig:L_posX_NB}. The profile is plotted at timestep 10,000, when the floating potential has stabilized. Note that the profile line does not pass through the centre of the octagonal body of the MMO, rather passing through one of the panels at an angle to the YZ plane, as will become apparent when the 2D potential profile is shown. Directly in front of the MMO, both configurations of the spacecraft show a distinct potential barrier forming on the photoemissive side. In \ref{fig:L_posX_WB} the potential difference 6.4 V floating potential of the spacecraft is $93.17 V$, and in \ref{fig:L_posX_NB} the height of the barrier relative to the plasma is $92.58 V$. The configurations of MMO with booms then has a barrier that is $0.59 V$ deeper than the configuration without booms. Both plots also show a small distinct drop in potential directly behind the spacecraft, likely caused by photoelectrons orbiting the craft before being reabsorbed.    


%AVERAGE POTENTIAL ISOLINES XY

\begin{figure}[H]
  \begin{subfigure}[b]{0.6\textwidth}
    \includegraphics[width=\textwidth]{figures/MMO/posX/WB/P_posX_WB.png}
    \caption{Booms}
    \label{fig:P_posX_WB}
  \end{subfigure}
  \begin{subfigure}[b]{0.6\textwidth}
    \includegraphics[width=\textwidth]{figures/MMO/posX/NB/P_posX_NB.png}
    \caption{Without booms}
    \label{fig:P_posX_NB}
  \end{subfigure}
  \caption{2D slices through $Z = 14.4 m$ showing the time averaged potential profile of the entire computational domain with drift along X axis, and photoemission included. The potential is time averaged after a floating potential has been reached after 1,000 timesteps.}
  \label{fig:Pot_posX}
\end{figure}

Figure \ref{fig:Pot_posX} plots contour lines for the potential in 2D slices of the computational domain. Contrasting figures \ref{fig:P_posX_WB} and \ref{fig:P_posX_NB} with figures \ref{fig:P_noPH_WB} and \ref{fig:P_noPH_NB}, we can see a thinner sheath form around both configurations around the MMO when photoemission is included. For both figures \ref{fig:P_posX_WB} and \ref{fig:P_posX_WB} we can also see that there are regions within the sheath on the sunlit side of the spacecraft where the potential differ varies along the booms. On both booms half way along the length, a small region of slightly higher potential, this suggests a gradient in the local electrical field. This variation in the potential could be the result of the angled geometry of the main body of the MMO, or potentially as a result of the voxelization of the spacecraft. Similarly, the sunlit panel of the octagonal main body closest to the sun shows a lower potential than the sunlit slanted panels. \cref{sec:appendixD} contains an analysis and discussion of these potential variations within the photoelectron sheath.

%PARTICLE DENSITIES (rho_i and rho_e)
%RHO_I

\begin{figure}[H]
  \begin{subfigure}[b]{0.6\textwidth}
  \includegraphics[width=\textwidth]{figures/MMO/posX/WB/I_posX_WB.png}
  \caption{Booms}
  \label{fig:I_posX_WB}
  \end{subfigure}
\begin{subfigure}[b]{0.6\textwidth}
  \includegraphics[width=\textwidth]{figures/MMO/posX/NB/I_posX_NB.png}
  \caption{Without booms}
  \label{fig:I_posX_NB}
\end{subfigure}
\caption{Ion density profile plotted at $Z = 14.4 m$, the color gradient is normalized against the ion plasma density from table \ref{tab:PlasmaParamMMO}. Plasma drift is along the X axis, and photoemission is included}
\label{fig:Ions_posX}
\end{figure}


%RHO_E

\begin{figure}[H]
  \begin{subfigure}[b]{0.6\textwidth}
  \includegraphics[width=\textwidth]{figures/MMO/posX/WB/E_posX_WB.png}
  \caption{Booms}
  \label{fig:E_posX_WB}
\end{subfigure}
\begin{subfigure}[b]{0.6\textwidth}
  \includegraphics[width=\textwidth]{figures/MMO/posX/NB/E_posX_NB.png}
  \caption{Without booms}
  \label{fig:E_posX_NB}
\end{subfigure}
\caption{Electron density profile plotted at $Z = 14.4 m$, the color gradient is normalized against the electron plasma density from table \ref{tab:PlasmaParamMMO}. Drift is along the X axis, and photoemission is included, the sun is located in the negative X direction.}
\label{fig:Electrons_posX}
\end{figure}


Plasma particle densities are plotted for both ions (protons) and electrons in figures \ref{fig:Ions_posX} and \ref{fig:Electrons_posX} respectively. Due to a high photoemission current, the positively charged spacecraft does not form an ion wake, as can be seen in both \ref{fig:I_posX_WB} and \ref{fig:I_posX_WB}. Downstream of the spacecraft, there is a relative low ion density relative to the ambient plasma which is more prominent in the case of the boomless configuration of the MMO. 

Variations in the ion density in the wake of a spacecraft is also realted to the plasma Mach number, as seen in the paper by Miloch et al. \insertref{Effects of booms of sounding rockets in flowing
plasmas, Miloch}. We can compute the Mach number of the plasma using equation \eqref{eq:ionMach}, and the Mach number equation
\begin{equation}
    M = v_d / c_s,
\end{equation}
where $v_d$ is the velocity of the plasma relative to the spacecraft. Thus far in our simulation we have assumed the relative velocity of the orbital speed of the MMO to be negligible as compared to the drift velocity of the plasma. However, to accurately compute the Mach number, we must compute the relative speed of the plasma to the MMO. Due to its eccentric orbit, the orbital speed of the MMO is computed using the equation
\begin{equation}\label{eq:orbSpeed}
    v = \sqrt{\mu \left(\frac{2}{r} - \frac{1}{a}\right)}.
\end{equation}
In equation \eqref{eq:orbSpeed}, $r$ denotes the distance from the MMO to the center of Mass of Mercury, $\mu$ is the standard gravitational parameter, and $a$ is the major axis of the elliptical orbit. The distance $r$ is approximately half the radius of Mercury at the location in the orbit selected for our simulations \parencite{Benna2009}, and the major axis can be found from the perihelion distance of 400 km, and aphelion distance of 11824 km of the MMO in orbit. For mercury, the gravitational parameter is $\mu = G M_{Mercury} = 2.2032 \times 10^{13}$. Substituting into equation \eqref{eq:orbSpeed}, the orbital speed of the MMO is 5.7 km/s. The highest possible speed of the plasma relative to the MMO is then 105.7 km/s. With the ion speed of sound $c_s$ we computed in the the previous section for charging without photoemission, the maximum Mach number possible in our simulation is $M \approx 1.08$. Since the Mach number is weakly supersonic, the ion enhanced wake is narrow and does not extend far from the spacecraft in the direction of the drift, which is in good agreement with results found by Paulsson et al. \parencite{Paulsson2019}.   

Figures \ref{fig:E_posX_WB} and \ref{fig:E_posX_NB} show high concentration of electrons adjacent to the sunlit surfaces of the MMO, with electron densities as high as 1175 times the density of electrons in the ambient plasma. 

At such high densities, the local Debye length is significantly shorter than that of the ambient plasma. We can compute the local Debye length from equation \eqref{eq:debyelength} and from the photoelectron temperature. The photoelectron temperature is computed in PINC internally by numerical integration of equation \eqref{eq:PlanckSum} in terms of energy and photons per second, subtracting the work function of the surface material gives the photoelectron temperature of $T_{ph} = 0.62 \; eV$. Substituting the photoelectron density and temperature into equation \eqref{eq:debyelength} we have a local Debye length of only 0.017 meters. Since our grid resolution is 0.225 meters, we are unable to resolve the local small scale variations in electrical potential due to Debye shielding.

We can estimate the thickness of the photoelectron sheath using the equation \parencite{Zhao1996}
\begin{equation}\label{eq:PhSheathThickness}
    n_{ph}(x_{ph}) = n_{e0}
\end{equation}
Where $n_{ph}(x_{ph})$ is the density of photoelectrons at a distance $x_{ph}$ from the spacecraft, and $n_{e0}$ is the electron density in the ambient plasma. The distance $x_{ph}$ is the radial distance from the spacecraft, and is normalized by the spacecraft radius $R$. Note that the implementation of photoelectron injection in this thesis does not separate photoelectrons as a separate species from the ambient plasma, and the distance between density measurements is relatively high (one cell width being 0.225 meters). By plotting the electron density as a line through the spacecraft parallel to the X axis, we estimate the photoelectron sheath to be 6.75 meters thick.


\subsection*{Drift parallel to Z axis}
Drift along the negative Z axis, with photoemission, and no external electric field (cases 3 and 8) is a more accurate plasma flow conditions that the MMO will experience in its orbit over the ecliptic north pole as opposed to drift along the X axis (cases2 and 7). Results for cases 3 and 8 will be compared to previous results shown for cases  2 and 7 to discern what effect, if any, drift direction has on the photoelectron barrier, the floating potential, and the plasma sheath. 

\begin{figure}[H]
  \centering
  \begin{subfigure}[b]{0.75\textwidth}
  \includegraphics[width=\columnwidth]{figures/MMO/minZ/WB/C_minZ_WB.png}
  \caption{Booms}
  \label{fig:C_minZ_WB}
\end{subfigure}
\par\bigskip
\begin{subfigure}[b]{0.75\textwidth}
  \includegraphics[width=\columnwidth]{figures/MMO/minZ/NB/C_minZ_NB.png}
  \caption{Without booms}
  \label{fig:C_minZ_NB}
\end{subfigure}
\caption{Time series plot of the potential of the MMO with and without booms, where the drift is along the negative Z axis and photoemission is included. The inset plots the same timeseries after 1000 timesteps, where the potential of the spacecraft has begun to oscillate about the floating potential.}
\label{fig:Conv_minZ}
\end{figure}

Figures \ref{fig:C_minZ_WB} and \ref{fig:C_minZ_NB} show the potential convergence to the floating potential for both MMO configurations. Comparing these figures to \ref{fig:C_posX_WB} and \ref{fig:C_posX_WB}, we can see that the final floating potential, and total charging time to the floating potential, seem to be independent of the drift direction. In the configuration of MMO with booms \ref{fig:C_posX_WB}, the potential converges to 105.42 V, and in the boomless configuration \ref{fig:C_minZ_NB} converges to 100.71 V. The difference in floating potential for the two directions of drift are so small as to be negligible. Since the photoemission current, and the emissive surfaces remain the same, only the ambient electron charging could be the cause of a difference in floating potential. 

The MMO without booms can be approximated as a cylinder, equation \eqref{eq:ICylinder} then gives the approximate current due to impining electrons. The current collected when the spacecraft is neutrally charged, $I_{th,e}$ is dependent on the cylinder geometry, from equation \eqref{eq:IthCyl}, it is apparent then that the current from the ambient electrons is dependent on the total surface area of the spacecraft and the magnitude of the drift velocity, but not on its direction.

%POTENTIAL THROUGH CENTER OF OBJECT
\begin{figure}[H]
  \begin{subfigure}[b]{0.6\textwidth}
  \includegraphics[width=\textwidth]{figures/MMO/minZ/WB/L_minZ_WB.png}
  \caption{Booms}
  \label{fig:L_minZ_WB}
\end{subfigure}
\begin{subfigure}[b]{0.6\textwidth}
  \includegraphics[width=\textwidth]{figures/MMO/minZ/NB/L_minZ_NB.png}
  \caption{Without booms}
  \label{fig:L_minZ_NB}
\end{subfigure}
\caption{Potential profile along the X axis for the two MMO configurations with drift along the negative Z axis and photoemission included. The line is plotted at $(x,y) = (13.95 m, 13.95 m)$, or node points $(x,y) = (62,62)$, and passes through the main octagonal body of the spacecraft. The X axis units are in number of nodes from the origin. The two values in each plot show the height of the potential barrier formed relative to the ambient plasma.}
\label{fig:Line_minZ}
\end{figure}

Figures \ref{fig:L_minZ_NB} and \ref{fig:L_minZ_WB} show the potential profile along the X axis for cases 3 and 8. The minimum potential in the potential barrier is slightly lower than for cases 2 and 6, the potential barrier height from figure \ref{fig:L_minZ_WB} is $093.33 V$, and for the MMO without booms figure \ref{fig:L_minZ_NB} gives a barrier height of $92.81 V$.

%AVERAGE POTENTIAL ISOLINES XY
\begin{figure}[H]
  \begin{subfigure}[b]{0.6\textwidth}
    \includegraphics[width=\textwidth]{figures/MMO/minZ/WB/P_minZ_WB.png}
    \caption{Booms}
    \label{fig:P_minZ_WB}
  \end{subfigure}
  \hfill
  \begin{subfigure}[b]{0.6\textwidth}
    \includegraphics[width=\textwidth]{figures/MMO/minZ/NB/P_minZ_NB.png}
    \caption{Without booms}
    \label{fig:P_minZ_NB}
  \end{subfigure}
  \caption{2D slices through $Z = 14.4 m$ showing the time averaged potential profile of the entire computational domain with drift along the negative Z axis, and photoemission included. The potential is time averaged after a floating potential has been reached after 1,000 timesteps.}
\label{fig:Pot_minZ}
\end{figure}


Figure \ref{fig:Pot_minZ} the 2D potential profile for simulation cases 3 and 8; no significant difference can be seen in the shape and thickness of the sheath formed around the MMO. Both figures \ref{fig:P_minZ_WB} and \ref{fig:P_minZ_NB} again show the distinct variation in potential within the sheath along the booms, and on the sunlit octagonal panel perpendicular to the Sun/MMO vector. 

%PARTICLE DENSITIES (rho_i and rho_e)
%RHO_I
\begin{figure}[H]
  \begin{subfigure}[b]{0.6\textwidth}
  \includegraphics[width=\textwidth]{figures/MMO/minZ/WB/I_minZ_WB.png}
  \caption{Booms}
  \label{fig:I_minZ_WB}
\end{subfigure}
\begin{subfigure}[b]{0.6\textwidth}
  \includegraphics[width=\textwidth]{figures/MMO/minZ/NB/I_minZ_NB.png}
  \caption{Without booms}
  \label{fig:I_minZ_NB}
\end{subfigure}
\caption{Ion density profile plotted at $X = 14.4 m$, the color gradient is normalized against the ion plasma density from table \ref{tab:PlasmaParamMMO}. Drift is along the negative Z axis. and photoemission is included.}
\label{fig:Ion_minZ}
\end{figure}

%RHO_E
\begin{figure}[H]
  \begin{subfigure}[b]{0.6\textwidth}
  \includegraphics[width=\textwidth]{figures/MMO/minZ/WB/E_minZ_WB.png}
  \caption{Booms}
  \label{fig:E_minZ_WB}
\end{subfigure}
\begin{subfigure}[b]{0.6\textwidth}
  \includegraphics[width=\textwidth]{figures/MMO/minZ/NB/E_minZ_NB.png}
  \caption{Without booms}
  \label{fig:E_minZ_NB}
\end{subfigure}
\caption{Electron density profile plotted at $Z = 14.4 m$, the color gradient is normalized against the electron plasma density from table \ref{tab:PlasmaParamMMO}. Drift is directed into the page, and photoemission is included. The sun is located in the negative X direction.}
\label{fig:Electron_minZ}
\end{figure}

The ion density and electron density contour plots for simulation cases 3 and 8 are given by figure \ref{fig:Ion_minZ} and figure \ref{fig:Electron_minZ} respectively, since drift is along the negative Z axis the ion densities are plotted in the YZ plane. Similarly to the ion densities for cases 2 and 7 both MMO configurations show a thin ion enhanced wake downstream of the spacecraft.


\section{Charging in an external magnetic field}

%FLOATING POTENTIAL CONVERGENCE
\begin{figure}[H]
  \begin{subfigure}[b]{0.75\textwidth}
  \includegraphics[width=\columnwidth]{figures/MMO/BField/WB/C_BField_WB.png}
  \caption{Booms}
  \label{fig:C_BField_WB}
\end{subfigure}
\par\bigskip
\begin{subfigure}[b]{0.75\textwidth}
  \includegraphics[width=\columnwidth]{figures/MMO/BField/NB/C_BField_NB.png}
  \caption{Without booms}
  \label{fig:C_BField_NB}
\end{subfigure}
\caption{Timeseries plot of the potential of the two configurations of the MMO, drift is along the negative Z axis, photoemission and an external magnetic field are included. The potential has been converted from PINC dimensionless units to Volts. The inset plots shows the potential of the two configurations for last 9,000 timesteps.}
\label{fig:Conv_BField}
\end{figure}

The convergence of the potential of the two MMO configurations to a floating potential for simulate case 4 and 9 are given in figure \ref{fig:Conv_BField}. Similarly to case 3 and case 9, the plasma drift is along the minus Z axis of the domain. It is apparent from both figures \ref{fig:C_BField_WB} and \ref{fig:C_BField_NB}, that the potential of the two MMO configurations converge to almost exactly the same floating potential as for the two simulations without an external field: $105.4 V$ for the MMO with booms, and $100.7 V$ in the boomless configuration. The same overshoot behaviour of the potential before convergence as seen in figure \ref{fig:C_BField_WB}, is also seen in figure \ref{fig:C_minZ_WB}. Figure \ref{fig:C_BField_NB} also displays the same behaviour of convergence to the floating potential as in figure \ref{fig:C_minZ_NB} without overshooting the floating potential.

Particles that enter the computational domain experience a force given by equation \eqref{eq:lorentz}, in the absence of an external electric field, this equation becomes
\begin{equation}
    \vb{F} = q\left(\vb{v} \cross \vb{B}\right) = - q \abs{v} \abs{B} \sin{\theta} \; \vb{j}.
\end{equation}
Here, we have used $\theta$ to denote the angle between the velocity vector and the magnetic field vector in the XZ plane, and $\vb{j}$ as the unit vector along the Y axis. Plugging in the drift velocity, the magnetic field strength, and $\theta = \ang{5.03}$  
a force of $1.404 \times 10^{-22} N$ will act on the proton and electron in the negative $\vb{j}$ and positive $\vb{j}$ direction respectively. This force will cause the ions and electron to spiral in opposite direction along guiding centers aligned with the magnetic field lines. We have already seen that drift direction does not affect the floating potential obtained from previous results, thus the floating potential remains the same. So called E cross B drift found close to the MMO where the electrical field is non-zero, is discussed in more detail in appendix \ref{sec:appendixD}. 
%POTENTIAL THROUGH CENTER OF OBJECT

\begin{figure}[H]
  \begin{subfigure}[b]{0.6\textwidth}
  \includegraphics[width=\textwidth]{figures/MMO/BField/WB/L_BField_WB.png}
  \caption{Booms}
  \label{fig:L_BField_WB}
\end{subfigure}
\begin{subfigure}[b]{0.6\textwidth}
  \includegraphics[width=\textwidth]{figures/MMO/BField/NB/L_BField_NB.png}
  \caption{Without booms}
  \label{fig:L_BField_NB}
\end{subfigure}
\caption{Potential profile along the X axis for the two MMO configurations with drift along the negative Z axis, photoemission and an external magnetic field are included. The line is plotted at $(x,y) = (13.95 m, 13.95 m)$, or node points $(x,y) = (62,62)$, and passes through the main octagonal body of the spacecraft. The X axis units are in number of nodes from the origin. The two values in each plot show the height of the potential barrier formed relative to the ambient plasma.}
\label{fig:Line_BField}
\end{figure}

%AVERAGE POTENTIAL ISOLINES XY
\begin{figure}[H]
  \begin{subfigure}[b]{0.6\textwidth}
    \includegraphics[width=\textwidth]{figures/MMO/BField/WB/P_BField_WB.png}
    \caption{Booms}
    \label{fig:P_BField_WB}
  \end{subfigure}
  \hfill
  \begin{subfigure}[b]{0.6\textwidth}
    \includegraphics[width=\textwidth]{figures/MMO/BField/NB/P_BField_NB.png}
    \caption{Without booms}
    \label{fig:P_BField_NB}
  \end{subfigure}
  \caption{2D slices through $Z = 14.4 m$ showing the time averaged potential profile of the entire computational domain with drift along the negative Z axis, photoemission and an external magnetic field are included. The potential is time averaged after a floating potential has been reached after 1,000 timesteps.}
  \label{fig:Pot_BField}
\end{figure}

Figures \ref{fig:L_BField_WB} and \ref{fig:L_BField_NB} show the potential profile for simulation cases 4 and 9 respectively. The potential barrier height with boom configuration of the MMO is $93.33 \; V$, and the barrier height for the configuration without booms is $92.81 \; V$. These potential barrier heights are the exact same as in simulation cases 3 and 8, with very small differences in floating potential and the minimum potential of the barrier: the difference in potential between the minimum potential in the barrier between case 8 and case 9 is only $0.02 \; V$.

Figure \ref{fig:Pot_BField} shows the 2D potential profile for cases 4 and 9. The overall structure of the potential sheath in both configurations of the MMO are similar to those found in cases 3 and 8. From figure \ref{fig:P_BField_WB}, the thickness in the sheath on the sunward side is however slightly larger than the sheath thickness seen in figure \ref{fig:P_minZ_WB}.


%PARTICLE DENSITIES (rho_i and rho_e)
%RHO_I
\begin{figure}[H]
  \begin{subfigure}[b]{0.6\textwidth}
  \includegraphics[width=\textwidth]{figures/MMO/BField/WB/I_BField_WB.png}
  \caption{Booms}
  \label{fig:I_BField_WB}
\end{subfigure}
\begin{subfigure}[b]{0.6\textwidth}
  \includegraphics[width=\textwidth]{figures/MMO/BField/NB/I_BField_NB.png}
  \caption{Without booms}
  \label{fig:I_BField_NB}
\end{subfigure}
\caption{Ion density profile plotted at $X = 14.4 m$, the color gradient is normalized against the ion plasma density from table \ref{tab:PlasmaParamMMO}. Drift is along the negative Z axis, photoemission and an external magnetic field are included.}
\label{fig:Ions_BField}
\end{figure}

%RHO_E
\begin{figure}[H]
  \begin{subfigure}[b]{0.6\textwidth}
  \includegraphics[width=\textwidth]{figures/MMO/BField/WB/E_BField_WB.png}
  \caption{Booms}
  \label{fig:E_BField_WB}
\end{subfigure}
\begin{subfigure}[b]{0.6\textwidth}
  \includegraphics[width=\textwidth]{figures/MMO/BField/NB/E_BField_NB.png}
  \caption{Without booms}
  \label{fig:E_BField_NB}
\end{subfigure}
\caption{Electron density profile plotted at $Z = 14.4 m$, the color gradient is normalized against the electron plasma density from table \ref{tab:PlasmaParamMMO}.Drift is along the negative Z axis, photoemission and an external magnetic field are included. The direction of the sun is along the negative X axis.}
\label{fig:Electrons_BField}
\end{figure}

Figures \ref{fig:Ions_BField} and \ref{fig:Electrons_BField} show the ion and electron density profile for both cases 4 and 9 respectively. The enhanced wakes formed downstream in both figure \ref{fig:E_BField_WB} and \ref{fig:E_BField_NB} show no discernible difference than the wakes formed in simulation cases 3 and 8, with similar shape and characteristic length. Likewise, there is no difference density of electrons on the photoemissive surfaces of the MMO, with the maximum density from \ref{fig:E_BField_WB} and \ref{fig:E_minZ_WB} being 1164.73 times the density of the electrons in the ambient plasma. 


\section{Charging at different photoelectron temperatures}
Internally in PINC, we have made the assumption of constant yield of photoelectrons by averaging the yield over the energy of incoming photoelectrons. Since our computation of the temperature is based on this averaging, another simulation was carried out where the photoelectron temperature was set to a constant, $T_{ph} = 3 \; eV$. The photoemission current was kept constant and no external magnetic field was included. Since drift was along the negative Z axis, these simulation cases (cases 5 and 10) are compared against simulation cases 3 and 8.  


%FLOATING POTENTIAL CONVERGENCE
\begin{figure}[H]
  \centering
  \begin{subfigure}[b]{0.75\textwidth}
  \includegraphics[width=\columnwidth]{figures/MMO/PHTemp/WB/C_PHTemp_WB.png}
  \caption{Booms}
  \label{fig:C_PHTemp_WB}
\end{subfigure}
\par\bigskip
\begin{subfigure}[b]{0.75\textwidth}
  \includegraphics[width=\columnwidth]{figures/MMO/PHTemp/NB/C_PHTemp_NB.png}
  \caption{Without booms}
  \label{fig:C_PHTemp_NB}
\end{subfigure}
\caption{Timeseries plot of the potential of the two configurations of the MMO, drift is along the negative Z axis and the photoelectron temperature has been set to $3 \; eV$. The potential has been converted from PINC dimensionless units to Volts. The inset plots the same timeseries after 1000 timesteps, where the potential of the spacecraft has begun to oscillate about the floating potential.}
\label{fig:Conv_PHTemp}
\end{figure}

Figures \ref{fig:C_PHTemp_WB} and \ref{fig:C_PHTemp_NB} show the converge of the potential convergence of two boom configurations for case 5 and 10. A significantly lower floating potential is reached for both boom configurations as compared to cases 3 and 8. The floating potential of the MMO with deployed booms was $78.15 \; V$, and $75.75 V$ for the MMO without deployed booms. The difference in floating potential between the two configurations is smaller than for the cases with a lower photoelectron temperature. Unlike case 3, figure \ref{fig:C_PHTemp_NB} shows an overshoot of the spacecraft potential before converging to the final floating potential, which could be due to a greater fraction of the photoelectrons having high enough kinetic energy to escape the potential barrier formed, thereby reducing the effective photoemission current. 

%POTENTIAL THROUGH CENTER OF OBJECT
\begin{figure}[H]
  \begin{subfigure}[b]{0.5\textwidth}
  \includegraphics[width=\textwidth]{figures/MMO/PHTemp/WB/L_PHTemp_WB.png}
  \caption{Booms}
  \label{fig:L_PHTemp_WB}
\end{subfigure}
\begin{subfigure}[b]{0.5\textwidth}
  \includegraphics[width=\textwidth]{figures/MMO/PHTemp/NB/L_PHTemp_NB.png}
  \caption{Without booms}
  \label{fig:L_PHTemp_NB}
\end{subfigure}
\caption{Potential profile along the X axis for the two MMO configurations with drift along the negative Z axis, photoemission is included with a photoelectron temperature of $3 \; eV$. The line is plotted at $(x,y) = (13.95 m, 13.95 m)$, or node points $(x,y) = (62,62)$, and passes through the main octagonal body of the spacecraft. The X axis units are in number of nodes from the origin. The two values in each plot show the height of the potential barrier formed relative to the ambient plasma.}
\label{fig:Line_PHTemp}
\end{figure}

%AVERAGE POTENTIAL ISOLINES XY
\begin{figure}[H]
  \begin{subfigure}[b]{0.6\textwidth}
    \includegraphics[width=\textwidth]{figures/MMO/PHTemp/WB/P_PHTemp_WB.png}
    \caption{Booms}
    \label{fig:P_PHTemp_WB}
  \end{subfigure}
  \hfill
  \begin{subfigure}[b]{0.6\textwidth}
    \includegraphics[width=\textwidth]{figures/MMO/PHTemp/NB/P_PHTemp_NB.png}
    \caption{Without booms}
    \label{fig:P_PHTemp_NB}
  \end{subfigure}
  \caption{2D cut through $Z = 14.4 m$ showing the time averaged potential profile of the entire computational domain with drift along the negative Z axis, photoemission is included with a photoelectron temperature of $3 \; eV$. The potential is time averaged after a floating potential has been reached after 1,000 timesteps.}
  \label{fig:Pot_PHTemp}
\end{figure}

Figure \ref{fig:Line_PHTemp} shows the potential profile of the MMO along the X axis for both boom configurations. In both cases, the minimum potential of the potential barrier is significantly lower than the minimum potential for other cases where the photoemission current is included: The barrier height is $85.58 \; V$ when booms are deployed and $85.23 \; V$ without booms. The absolute value of the potential barrier height for the simulation cases $T_{ph} = 3 \; eV$ is therefore substantially larger than for cases with $T_{ph} = 0.62 \; eV$. 

The 2D potential profile for case 10 and case 5 is shown by figure \ref{fig:P_PHTemp_WB} and figure \ref{fig:P_PHTemp_NB} respectively. The potential at the saddle point located adjacent to the sunlit surface perpendicular to the vector pointing towards the sun is noticeably more negative for both boom configurations as compared to cases 3 and 8. Additionally, the potential of the two symmetrical spikes along the booms in figure \ref{fig:P_PHTemp_WB} is lower than the spikes seen in figure \ref{fig:P_minZ_WB} for case 8.

%PARTICLE DENSITIES (rho_i and rho_e)
%RHO_I
\begin{figure}[H]
  \begin{subfigure}[b]{0.6\textwidth}
  \includegraphics[width=\textwidth]{figures/MMO/PHTemp/WB/I_PHTemp_WB.png}
  \caption{Booms}
  \label{fig:I_PHTemp_WB}
\end{subfigure}
\begin{subfigure}[b]{0.6\textwidth}
  \includegraphics[width=\textwidth]{figures/MMO/PHTemp/NB/I_PHTemp_NB.png}
  \caption{Without booms}
  \label{fig:I_PHTemp_NB}
\end{subfigure}
\caption{Ion density profile plotted at $Z = 14.4 m$, the color gradient is normalized against the ion plasma density from table \ref{tab:PlasmaParamMMO}. Drift is along the negative Z axis, and photoemission is included with a photoelectron temperature of $3 \; eV$.}
\label{fig:Ions_PHTemp}
\end{figure}

%RHO_E
\begin{figure}[H]
  \begin{subfigure}[b]{0.6\textwidth}
  \includegraphics[width=\textwidth]{figures/MMO/PHTemp/WB/E_PHTemp_WB.png}
  \caption{Booms}
  \label{fig:E_PHTemp_WB}
\end{subfigure}
\begin{subfigure}[b]{0.6\textwidth}
  \includegraphics[width=\textwidth]{figures/MMO/PHTemp/NB/E_PHTemp_NB.png}
  \caption{Without booms}
  \label{fig:E_PHTemp_NB}
\end{subfigure}
\caption{Electron density profile plotted at $Z = 14.4 m$, the color gradient is normalized against the electron plasma density from table \ref{tab:PlasmaParamMMO}. Drift is along the negative Z axis, and photoemission is included with a photoelectron temperature of $3 \; eV$.}
\label{fig:Elec_PHTemp}
\end{figure}


Figure \ref{fig:Ions_PHTemp} show a 2D profile of ion density for case 10 and case 5. Comparing figure \ref{fig:I_PHTemp_WB} with \ref{fig:I_PHTemp_WB}, we can see that the length of the wake formed behind the MMO in the direction of drift is shorter than in figure \ref{fig:I_minZ_WB} for case 3. If we extract the ion density for each cell in the Z axis at $X = 64$ and $Y = 64$, we can estimate the length of the wake formed applying the following equation
\begin{equation}
    n_i(z_w) = n_{i0}.
\end{equation}
Where $z_w$ denotes the length of the wake measured from the bottom panel of the MMO to the edge of the wake, and $n_{i0}$ denotes the density of the ambient ions in the plasma upstream from the MMO. Using this method, the length of the wake for case 10 is $4.73 \; m$, and $5.53 m$ for case 3.

The 2D electron density profile for case 10 and case 5 are given in figure \ref{fig:E_PHTemp_WB}. Although not immediately obvious from the color contours, the maximum electron density adjacent to sunlit surfaces is slightly higher in case 5 and 10 as compared to the maximum electron density in case 3 and 8. The maximum density of electrons for case 10 is 1180.45 times the ambient electron density, and 1164.74 times the ambient electron density for case 8. The following equation gives the density of photoelectrons able to escape the potential barrier \parencite{Zhao1996}
\begin{equation}
    n_{ph}(x) = \frac{J_{ph}(x)}{e \overline{v}_{ph}}.
\end{equation}
Where $J_{ph}$ has been used to denote the absolute value of the effective photoelectron current density, and $\overline{v}_{ph}$ denotes the mean velocity of emitted photoelectrons. The effective photoelectron current density is however proportional to the height of the potential barrier \parencite{Zhao1996}
\begin{equation}
    J_{ph} \propto \exp{- \frac{e \phi_b}{k T_{ph}}}.
\end{equation}
Where $\phi_b$ denotes the potential barrier height. Since the barrier height is much larger in case 8, the effective current density flux compensates for the higher average velocity of photoelectrons in case 10, leading to a higher maximum density of photoelectrons in case 10.

\section{Summary of results}
We can now summarize the results for all the simulation cases completed. For all the simulation cases, the floating potential of the MMO was computed, and an estimate for the plasma sheath thickness was found. In the simulation cases where the photoemission current was included, we also computed the potential barrier height due to the photoelectron cloud formed adjacent to the sunlit surfaces of the spacecraft. Table \ref{tab:MMOresultsSummary} shows these values for all simulation cases run for this thesis.

\begin{center}
\begin{table}[H]
%\centering
\begin{tabular}{p{2cm}|p{1.5cm}|p{1.5cm}|p{3cm}}
\toprule
\toprule
 Simulation case & Floating potential (V) & Barrier height (V) & sheath thickness(m)\\
\hline
Case 1 & $-283^{*}$ & N/A & 2.5\\
\hline
Case 2 & 100.70 & 92.58 & 6.75\\
\hline
Case 3 & 100.71 & 92.81 & 6.75\\
\hline
Case 4 & 100.70 & 92.81 & 6.3 \\
\hline
Case 5 & 75.75 & 85.23 & 4.95 \\
\hline
Case 6 & $-281^{*}$ & N/A & 3.2\\
\hline
Case 7 & 105.40 & 93.17 & 6.525 \\
\hline
Case 8 & 105.42 & 93.33 & 6.75 \\
\hline
Case 9 & 105.40 & 93.33 & 6.3 \\
\hline
Case 10 & 78.15 & 85.58 & 6.75 \\
\bottomrule
\bottomrule
\end{tabular}
\caption{Summary of computed floating potentials, barrier height for photoemissive simulation cases, and the plasma sheath thickness. The asterisk over the floating potential values for cases 1 and 6 indicate the values as estimations from the curve fitting given in \cref{sec:appendixC}}
\label{tab:MMOresultsSummary}
\end{table}
\end{center}