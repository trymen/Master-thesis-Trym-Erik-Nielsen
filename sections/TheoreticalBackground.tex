\chapter{Theoretical background}

\label{sec:theory}

\section{Plasma modelling}
In this section, we present a short overview of the basic mathematical framework used in the study of plasma dynamics. In section \cref{subsec:basicEq}, we briefly present the equations of motion for single particle motion in electrical and magnetic field. Then in \cref{subsec:pParam} we give an overview of essential plasma parameters.
Plasma is a collection of ionized gas, commonly referred to as the fourth state of matter. The solar wind, the polar aurorae, and lighting are some examples of plasmas that occur in nature. \insertref{ref a basic intro text} Plasma shares many of the same properties that describe gases, but differ in being affected by magnetic and electrical fields: since plasma consists of charged particles, ions and free electrons, they are subject to the Lorentz force. The Lorentz force acting on a charged plasma particle, causes curvilinear motion further complicated by the influence of other nearby charged particles.

\section{Basic equations}

\subsection{Single particle description}
\label{subsec:basicEq}
The single particle description of a plasma describes the motion of individual charged particles moving in imposed magnetic and electrical fields. Assuming the force of gravity is sufficiently small, and assuming constant electrical and magnetic fields, the single particle motion of a charge $q$ moving at velocity $v$ in the electrical field $E$ and magnetic field $B$ is described by the Lorentz force law 

\begin{equation}\label{eq:lorentz}
    \vb{F} = q \vb{E} + q \vb{v} \cp \vb{B}
\end{equation}

Where the term $q \vb{E}$ is called the Electric force, and the term $q\vb{v} \cp \vb{B}$ is called the magnetic force. When a particle moves in a static magnetic field, and no electric field is present, the particle will gyrate around magnetic field lines. Setting $E$ to zero, we have

\begin{equation}\label{eq:magF}
    \vb{F} = q \vb{v} \cp \vb{B}
\end{equation}

The cross product of the velocity and magnetic field vector means that the magnetic force always acts perpendicularly to the direction of motion of the particle, thereby causing the particle to gyrate.

Setting the centripetal force equal to the magnitude of the Lorentz force, we can derive an expression for the gyroradius $r_g$ of the motion

\begin{equation}\label{eq:gyrorad}
    \frac{m_s v_{\perp}^2}{r_g} = \abs{q} v_{\perp} \vb{B}
\end{equation}

Where the subscript on $m_s$ denotes the specie of the particle. and $V_\perp$ denotes the perpendicular component of velocity to  the plane of $B$. Upon rearranging, the expression for the gyroradius $r_g$ becomes

\begin{equation}
    r_g = \frac{m_s v_\perp}{\abs{q} \vb{B}}
\end{equation}

The particle gyrates with an angular frequency, called the cyclotron frequency $\Omega_c$, expressed as 

\begin{equation}
    \Omega_c = \frac{v_\perp}{r_g} = \frac{\abs{q} \vb{B}}{m_s}
\end{equation}

The motion of a particle $q_s$ is in practical cases often modelled as the drift of the centre of gyration of the particle. When also subjected to an isotropic electrical field, this motion is called E cross B drift, or Hall drift, and can be derived from the Lorentz force equation and Newtons' second law and solving for the acceleration of the particle: If we assume the drift velocity to be constant in time, the expression for the drift $\vb{V}_D$ becomes

\begin{equation}
    \vb{V}_D = \frac{\vb{E} \cp \vb{B}}{B^2}
\end{equation}

\subsection{Kinetic description}
In the previous section, we studied the individual motion of charged plasma particles. Although the single particle description of plasma is useful in gaining an understanding in how individual particles behave in isotropic magnetic and electrical fields; it is an impractical model for analyzing macroscopic phenomena of the plasma. \insertref{"Plasma Physics" by Alexander Piel contains a good discussion of when each plasma model should be used} The kinetic description of plasma begins with the assumption of a density distribution of charges in the six dimensional phase space varying over time. Let $f_s(\vb{x}, \vb{v}, t)$ be the continuous probability distribution, representing the probability of finding a charged particle of species $s$ at time $t$ in phase space. Multiplying the distribution of charges by the charge of the species $q_s$, and integrating over the velocity space. the charge density $\rho_s$ can be found. Summing over all the species of charge in the plasma gives us the following expression for the total charge density $\rho_c$

\begin{equation}\label{eq:rhoCharge}
    \rho_c = \sum_s q_s\int f(\vb{x}, \vb{v}, t) d^3\vb{v}
\end{equation}

Similarly, an expression for the current density is obtained by multiplying the charge distribution by the velocity vector $v$ and integrating the result in a similar fashion to \ref{eq:rhoCharge}

\begin{equation}
    \vb{j} = \sum_s q_s \int \vb{v} f(\vb{x}, \vb{v}, t) d^3 \vb{v}
\end{equation}


Equipped with a continuous distribution function of particles in phase-space, equations of motion that describe the flow of the charged particles can be derived from solving the Boltzmann equation. Or alternatively, when assuming a non-collisional plasma, the set of vector equations called the Vlasov equation can be used


\begin{equation}\label{eq:Vlasov}
    \pdv{f}{t} + \vb{v} \vdot \grad{f_s} + \frac{q_s}{m_s} (\vb{E} + \vb{v} \cp \vb{B}) \vdot \grad_v{f} = 0
\end{equation}

Where the operator notation $\grad_v = (\pdv{v_x}, \pdv{v_y}, \pdv{v_z})$ and $\grad = (\pdv{x},\pdv{y},\pdv{z})$ has been used. For a comprehensive derivation of the Vlasov equation see \insertref{Plasma physics via computer simulation for example}


\subsection{Fluid description}
In previous sections, the equations of motion for characterising plasma has been modelled by analyzing the forces acting on individual particles. While this approach can be helpful in gaining insight into the physics governing the plasma behaviour, it is difficult to apply these frameworks to practical computation models. 
\vskip 1mm
Another method, that reduces the complexity of computing individual particle motions, is treating plasma as two continuous fluids. In this approach we are able to extract macroscopic properties of the plasma, such as the density, the velocity and the mean energy. The fluid equations are derived by taking the velocity moments of the Vlasov equation \ref{eq:Vlasov}, where the generalized velocity moment is described as

\begin{equation}\label{eq:moment}
    M^n \equiv \int f(\vb{v}) \vb{v}^n d \vb{v}
\end{equation}

The zeroth velocity moment, also called the continuity equation, can be found by multiplying equation \ref{eq:Vlasov} by the zeroth velocity moment 

\begin{equation}\label{eq:zeromoment}
    n_s = \int f_s d \vb{v}
\end{equation}

The zeroth velocity moment for species $s$ is simply the number density for the species. Multiplying equation \ref{eq:zeromoment} by the Vlasov equation we have

\begin{equation}\label{eq:continuity}
    \int \pdv{f_s}{t} d \vb{v} \int \vb{v} \vdot \grad{f_s} \vb{v} + \frac{q}{m} \int \vb{E} + \vb{v} \cp \vb{B} \vdot \pdv{f_s}{\vb{v}} d \vb{v} = 0
\end{equation}

With a considerable amount of manipulation \insertref{Chen, ch 7}, equation \ref{eq:continuity} reduces to

\begin{equation}\label{eq:contReduced}
    \pdv{n_s}{t} + \div{(n_s \vb{v_s})} = 0 
\end{equation}


In a similar fashion, the momentum equation can be derived from multiplying equation \ref{eq:Vlasov} by the the first velocity moment

\begin{equation}\label{eq:firstmoment}
    M^1 = \int m_s \vb{v} f_s d \vb{v}
\end{equation}

this multiplication yields

\begin{equation}\label{eq:momentum}
     \pdv{\vb{m_s n_s v_s}}{t} + \div{\vb{P_s}} - e_s n_s (\vb{E} \ \vb{v_s} \cross \vb{B}) = \vb{F_s}
\end{equation}

Where $P_s$ denotes the pressure tensor field. The energy equation is derived from the second velocity moment

\begin{equation}
    M^2 = \int \frac{1}{2} m_s \vb{v} \vb{v} f_s d \vb{v}
\end{equation}

The second order moment described the flow of momentum, and is also called the stress tensor \insertref{Fitzpatrick, eq 3.17, plasma physics: an introduction}. Applying the second velocity moment to equation \ref{eq:Vlasov} equation gives the partial differential equation

\begin{equation}\label{eq:energy}
    \pdv{t} (\frac{3}{2} p_s + \frac{1}{2} m_s n_s v_s^2) + \div{\vb{Q_s}} - e_s n_s \vb{E} \vdot \vb{v_s} = W_s + \vb{v_s} + \vb{v_s} + \vb{F_s}   
\end{equation}

Where the terms $Q_s$ and $W_s$ denote the energy flux density and total change of energy of species $s$ respectively. Equations \ref{eq:contReduced}, \ref{eq:momentum}, and \ref{eq:energy}, are collectively known as the fluid equations, and written in their convective time-derivative form, can be reduced to

\begin{subequations}
    \begin{align}
        &\frac{D n_s}{D t} + n_s \div{\vb{v_s}} = 0 \\
        &m_s n_s \frac{D \vb{v_s}}{D t} + \div{p_s} - e_s n_s ( \vb{E} + \vb{v_s} \cross \vb{B}) = \vb{F_s} \\
        &\frac{3}{2} \frac{D p_s}{D t} + \frac{3}{2} p_s \div{v_s} + \vb{p_s} \colon \grad{\vb{v_s}} + \div{\vb{q_s}} = W_s
    \end{align}
\end{subequations}

Formal derivations of the fluid equations, with some differences in notation, can be found in the fundamental works of Chen \insertref{Chen 1985}, and Fitzpatrick \insertref{Fitzpatrick 2012}.

\subsection{Magnetohydrodynamics}
\todo{Finish this section after }

\section{Plasma parameters}\label{subsec:pParam}
In this section, the basic parameters used in the analysis and simulation will be introduced. We discuss the notion of a plasma temperature, present the characteristic length and frequency scales the Debye length, and plasma frequency. Additionally, we discuss the principle of quasineutrality in relation to plasma. 

\subsection{Temperature}\label{subsec:temperature}
The temperature of a plasma is a measure of the average kinetic energy of the individual species that make up the plasma. For a mono-atomic gas, with a probability distribution $f(u)$, Chen \insertref{Chen, section 1.3} gives the following equation for the average kinetic energy

\begin{equation}\label{eq:avgEint}
    E_{av} = \frac{\int^\infty_{-\infty} \frac{1}{2} m u^2 f(u) du}{\int^\infty_{-\infty} f(u) du}
\end{equation}


By defining the variables 

\begin{equation*}
    v_{th} \equiv (2 K T / m)^{\frac{1}{2}} \hspace{10pt} \text{and} \hspace{10pt} y \equiv \frac{v_{th}}{u}
\end{equation*}
    
and integrating the numerator by parts, equation \ref{eq:avgEint} reduces to

\begin{equation}\label{eq:avgE1D}
    E_{av} = \frac{1}{2} K T
\end{equation}

Where $K$ denotes the Boltzmann constant. Chen extends this argument for particles with three degrees of freedom. Integrating Maxwell's distribution in three dimensions in a similar fashion to \ref{eq:avgEint} and cancelling the integrals, the average kinetic energy is

\begin{equation}\label{eq:avgE3D}
    E_{av} = \frac{3}{2} K T
\end{equation}

In plasma physics, it is common to use average energy rather than temperature as they are so closely linked. Calculations are often reduced in complexity by using the electron volt as the basic units of energy. It is defined as the amount of energy required to move a single electron across an electrical potential difference of 1 volt. One eV is therefore approximately equal $1.6 \text{x} 10^-19$ J, with a conversion factor to temperature, in Kelvin

\begin{equation*}
    1 eV \approx 11600 K
\end{equation*}

\subsection{Debye length}
The Debye length is a parameter that measures the net persistence of the electrostatic effect of a charged particle. This effect is also known as Debye shielding, Hutchinson \insertref{Hutchinson plasma diagnostics sections 3.1.2} presents a definition for the Debye length by analyzing the potential profile of a test charge placed in a cold plasma. In a non isothermal plasma ions can be assumed to be stationary in relation the more energetic electrons. The electrons density is then determined from the Boltzman factor

\begin{equation}\label{eq:boltzmannFactor}
    n_e = n_{\infty} \exp(\frac{e \phi}{T_e})
\end{equation}

Where $T_e$ denotes the electron temperature, $n_{\infty}$ is the density of electrons far away from the perturbing test charge, and where $\phi$ is the electrical potential as a function of the radial distance from the test function. Inserting equation \ref{eq:boltzmannFactor} into the Poisson's equation;

\begin{equation}\label{eq:poissonDebye}
    \laplacian{\phi}  = - \frac{\rho}{\epsilon_0} = \frac{-e}{\epsilon_0} (n_i - n_e) = \frac{-e}{\epsilon_0} n_{\infty} \left[1 - \exp (\frac{e \phi}{T_e}) \right]
\end{equation}

Assuming $e \phi \lll T_e$, then it is reasonable to only keep the linear terms of the Taylor expansion of the $\exp (\frac{e \phi}{T_e})$ term on the right hand side of equation \ref{eq:poissonDebye}. Poisson's equation then takes the form

\begin{equation}
    \laplacian{\phi} - \frac{1}{\lambda_D^2} \phi = 0
\end{equation}


Where the Debye length $\lambda_D$ has been defined as follows:

\begin{equation}\label{eq:Helmholtz}
    \lambda_d \equiv \sqrt{\frac{\epsilon_0 k T_e}{n_e e^2}}
\end{equation}

The solution of equation \ref{eq:Helmholtz} takes the form

\begin{equation*}
    V \propto \exp (\frac{\pm x}{\lambda_D})
\end{equation*}

The Debye length is then a measure of the shielding distance or thickness of the sheath that forms around a charged object embedded in a plasma \insertref{Chen section 1.4}.


\subsection{quasineutrality}
Now that we have presented an expression for the Debye length, the approximation of quasineutral plasma can be defined: When the length scale of interest is much smaller than the Debye length, i.e $\lambda_D \lll L$, the density of ions is approximately equal to the density of electrons\insertref{Chen, section 1.4 Debye Shielding}. Or in other terms

\begin{equation}
    n_i \approx n_e \approx n
\end{equation}

Where n is a common density called the plasma density \insertref{Chen, section 1.4, Debye shielding}. When this approximation holds, the plasma is said to be quasineutral. Note that this approximation still allows for variations in charge at smaller length scales than $\lambda_D$. 

\subsection{Electron plasma frequency}
In plasma simulations of the solar wind an important physical phenomena is plasma oscillation, otherwise called Langmuir waves. In a neutral plasma, disturbances in the density of electrons (and ions) cause oscillations due to the restorative Coulomb force. For solar wind, and other cold plasmas, the frequency of the oscillations of electrons are expressed as

\begin{equation}\label{eq:plasmaFreq}
    \omega_{pe} = \sqrt{\frac{n_e e^2}{\epsilon_0 m_e}}
\end{equation}

$\omega_{pe}$ is called the electron plasma frequency. A similar expression exists for the oscillations of ions, but due to their higher mass they oscillate much slower rate than the electrons. Equation \ref{eq:plasmaFreq} is formally derived from Maxwell's equations \insertref{derivations by Chen sec 4.3, Fitzpatrick, Hutchinson} by assuming an infinite plasma not affected by an external magnetic field, no thermal motion i.e $KT = 0$, and ion fixed in space  in a uniform distribution \insertref{Chen sec 4.3}.



\section{The solar wind plasma environment}
In this section, a brief overview of the physics and basic plasma parameter values of the solar wind are presented. These data will necessary in building the computational parameters required in the charging simulations of space probes moving in the interplanetary medium presented later in this thesis.
\\
The solar wind is a non-isothermal plasma with a temperature averaging around 10 $eV$, that is primarily composed of free electrons, ions and alpha particles. The density of the solar wind plasma is relatively low as compared to other naturally occurring plasmas with a mean value of around 5 $cm^{-3}$. It flows with a drift velocity ranging from 400 $km/s$ up to 900 $km/s$ during periods of high solar activity, \insertref{fundamentals of spacecraft charing, prologue}. In close proximity to the sun, the solar wind density increases significantly, potentially charging space probes to values as low as hundreds of volts \insertref{Deca 2013, and introduction to spacecraft charging (?)}.


\todo{Check if I am allowed to use an image from copyrighted works!}
%\begin{figure}[h!]
%    \centering
%    \includegraphics[scale=0.65]{figures/solarWind.png}
%    \caption{Large-scale structure of the solar wind}
%    \label{fig:solarWind}
%\end{figure}




\subsection*{Interactions with solar system objects}
The properties of the solar wind can change significantly in regions in close proximity to objects. Planets with internal magnetic fields, will cause MHD \todo{can I use abbreviations?} shock formation \insertref{one of the magnetosphere papers?}, and even bodies without an internal magnetic field of their own can 

\begin{equation}\label{eq:magFieldStrength}
    B_{\mu} = (\mu_0 / 4 \pi) \mu / r^3_M
\end{equation}



\begin{equation}\label{eq:pressureBalance}
    \rho_w v^2_w \approx \frac{(2 B_{\mu})^2}{2 \mu_0} 
\end{equation}


\begin{equation}
    r_M \approx \left(\frac{\mu}{4 \pi} \sqrt{\frac{2 \mu_0}{\rho_w v^2_w}} \right)^{\frac{1}{3}}
\end{equation}

\section{Spacecraft charging in plasma}
A spacecraft collects charged particles when moving relatively to ambient plasma, this change in net charge on the spacecraft causes an electrical field according to Gauss's Law \insertref{Fundamentals of SC charging ch 1}. When the charging of the spacecraft reaches a non-oscillatory steady state, the sum of currents from all the species in the plasma is zero, and the net charging is constant. This is expressed as 

\begin{equation} \label{eq:currentBalance}
    \dv{Q}{t} = \sum_s I_s(V) = 0
\end{equation}

The solution of equation \ref{eq:currentBalance} for $V$ is called the floating potential of the spacecraft, and is relative to the neutral plasma. The floating potential of a spacecraft in a drifting plasma is usually negative, since the electrons move at a higher speed than the ions.
\\
For complex geometries it is non-trivial to find analytical expressions for the currents $I_s$ charging the spacecraft; in this section, Langmuir probe theory and orbital motion-limited theory is discussed as the basic tools for analyzing spacecraft charging with simple geometries. Spacecraft charging in Maxwellian plasmas, and the primary charging mechanisms that effect the overall potential of a spacecraft are then presented. Special attention is given to the photoelectric effect and the spacecraft photoelectric current, as the implementation of this current in a computational model is the overall objective of this thesis \todo{clunky should I mention this}.


\subsection{Langmuir probe theory}
A Langmuir probe is a simple device used in the laboratory to measure the temperature, the density, and the electrical potential of a plasma. A Langmuir probe consists of one or more, often spherical, electrodes inserted into the ambient plasma. Either a time varying or constant potential is induced between the electrodes, or between the electrodes and some electrical ground, causing the probe to collect charged particles. By analysing the current $I(V)$ for different potentials, the properties of the surrounding plasma can be extracted \insertref{Marholm Ph.D ch 3, fundamentals of S/C charging, Wojciech et al.}. When immersed in a plasma, a spacecraft behaves in a similar fashion to a Langmuir probe. The difference is that in the case of a spacecraft a potential is not applied, but can be measured as a response to incoming and outgoing currents \insertref{Fundamentals of sc charging, ch 2}. 


\subsection*{Orbital motion-limited theory}
In orbital motion limited theory, or OML theory for short, the charge flux impinging on a spacecraft is determined from the conservation of energy and angular momentum of species travelling in proximity to a spacecraft \insertref{Marholm Ph.D ch 3}. From conservation of energy, an expression for the impact velocity on a spherical spacecraft can be obtained, and from conservation of momentum an expression for the distance from the centre of the spacecraft to the straight line of travel can be found. In \insertref{Fundamentals of sc charging, Lai, Ch2.1} these expressions are given as 

\begin{subequations}
    \begin{align}
        v_a &= v_{th} \left(1 - \frac{q \phi}{kT} \right) \label{eq:impactvel} \\
        h &= a \left(1 - \frac{q \phi}{kT} \right) \label{eq:distStraight}
    \end{align}
\end{subequations}

Integrating the particle distribution $f_s(\vb{x}, \vb{v}, t)$ using equation \ref{eq:impactvel} and \ref{eq:distStraight} as limits of integration, we can find the flux of particles that impact the spacecraft. It is often convenient work with a normalized potential variable in the formulas for current, define

\begin{equation}\label{eq:normPot}
    \eta_s \equiv - \frac{q_s V}{k T_s}
\end{equation}

Using the normalized potential, particles with $n_s < 0$ will be repelled, and particles with $n_s > 0$ will be attracted to the spacecraft \insertref{Marholm Ph.d ch 3.1}. Expressions for the integration of the particle distribution has been found for some simple geometries, for species with $n_s > 0$ with the assumption of a plasma with no internal magnetic field, and zero drift by Mott-Smith and Langmuir \insertref{Mott-Smith and Langmuir}

\begin{subequations}
    \begin{align}
        I_s (\eta_s) &= I_{th,s} \label{eq:IPlane} \\
        I_s (\eta_s) &= I_{th,s} \left(\frac{2}{\sqrt{\pi}} \sqrt{\eta_s} + \exp(\eta_s) \hspace{2pt} \text{erfc}(\sqrt{\eta_s}) \right) \label{eq:ICylinder}\\
        & \approx I_{th,s} \frac{2}{\sqrt{\pi}} \sqrt{1 + \eta_s} \\
        I_s(\eta_s)  &= I_{th,s} (1 + \eta_s) \label{eq:Isphere}
    \end{align}
\end{subequations}

Where equations \ref{eq:IPlane}, \ref{eq:ICylinder} and \ref{eq:Isphere} are the functions of current for a plane, a cylinder, and a sphere respectively. These functions are derived based on the assumption that the cylinder is long compared to $\lambda_D$, the radius of the sphere and the cylinder to be small compared to $\lambda_D$, and the plane to be wider than $\lambda_D$. The notation $I_{th}$ is used to denote the current collected when the object is neutrally charged, i.e $\eta_s = 0$, and $erfc(\sqrt{\eta_s})$ is the complementary error function defined as

\begin{subequations}
    \begin{align*}
        \text{erfc}(z) &\equiv 1 - \text{erf}(z) \\
        &= \frac{2}{\sqrt{\pi}} \int^\infty_z e^(-t^2) dt
    \end{align*}
\end{subequations}


\subsection*{Spacecraft charging in a Maxwellian plasma}


\subsection*{The photoelectric effect}\todo{Includes discussion on Blackbody radiation..}

\subsection*{Secondary and backscattered electrons}

\subsection*{Artificial sources}
