\chapter{Abstract}
\label{sec:abstract}
This thesis presents results from plasma charging simulations of the Mercury Magnetospheric Orbiter (MMO) spacecraft orbiting Mercury in direct sunlight. These simulations were carried out using the Particle-In-Cell program PINC. The object-plasma interaction module of the PINC framework was extended by including a photoemission current algorithm based on the work by Cartwright et al. \parencite{Cartwright2000} and Deca et al. \parencite{Deca2013}. The new photoemission current model was tested rigorously by comparing the results of test cases run in other Particle-In-Cell codes, proving the models validity. The charging behaviour of the MMO spacecraft was then simulated for varying plasma conditions, and with the spacecraft booms extended and retracted. The results of these simulations are shown to agree with the spacecraft charging theories developed by Langmuir \parencite{Mott-Smith1926}, Whipple \parencite{Shipple1981} and \parencite{Garrett1981}. The MMO charging simulations are also compared with similar Particle-In-Cell simulations, and differences and similarities are discusssed.